\documentclass[12pt,a4paper,oneside]{book}
\usepackage[utf8]{inputenc}
\usepackage[document]{ragged2e}
\usepackage{fullpage}
\usepackage[margin=2cm]{geometry}
\usepackage{forest}
\usepackage[T1]{fontenc}
\usepackage{enumitem}
\usepackage{relsize}
\usepackage{amsmath}
\usepackage{amssymb}

\setcounter{MaxMatrixCols}{50}
\setcounter{section}{-1}

\renewcommand{\chaptername}{Capitolo}
\renewcommand{\contentsname}{Indice}

\begin{document}

%%%%%%%%%%%%	TITOLO, AUTORE E INDICE 	%%%%%%%%%%%%

\title{Riassunto di Algebra Lineare}
\author{Riccardo Montagnin e Leonardo Bragagnolo}
\date{}

\maketitle

\tableofcontents

%/////////////////////////////////////////////////////////////////////////////////////////////////////////


%%%%%%%%%%%% PRIMO CAPITOLO - SPAZI VETTORIALI %%%%%%%%%%%%

				\chapter{Spazi vettoriali}

			  %%%%%%	Teoria di base	%%%%%

				\section{Teoria di base}


\begin{itemize}

\item \textbf{Spazio vettoriale.} \linebreak
	  Uno spazio vettoriale su $\mathbb{K}$ campo è un insieme $V$ su cui sono definite due operazioni:

	  \begin{enumerate}
	  \item \textbf{Somma} (+)
	  \item \textbf{Prodotto per scalare} ($\cdot$)
	  \end{enumerate}

	  le quali restituiscono un risultato appartenente ancora a V

\item \textbf{Sottospazio vettoriale.}
	  \linebreak Un sottospazio vettoriale su $\mathbb{K}$ campo è un sottoinsieme $W$ di uno spazio 		
	  vettoriale chiuso rispetto la somma e il prodotto per scalare. Ovvero \break $\forall w_1, ..., w_n 
	  \in W$ e $\lambda \in \mathbb{K} \Rightarrow w_1+w_2 \in W$ e $\lambda w_n \in W$

\item \textbf{Span dei vettori $v_1, ..., v_n$}. \linebreak
	  Viene definito sottospazio generato dai vettori $v_1, ..., v_n$ l'insieme di tutte le possibili 
	  combinazioni lineari di $v_1, ..., v_n$. \hfill \break
	  Inoltre $Span(v_1, ..., v_n) = \{\alpha_1 v_1 + ... +\alpha_kv_n\}$.

\item \textbf{Vettori linearmente dipendenti}. \linebreak
	  Un insieme di vettori $v_1, ..., v_n$ si dice linearmente dipendente se esistono $\alpha_1, ..., 
	  \alpha_k$ non tutti nulli tali che $\alpha_1v_1, ..., \alpha_nv_n = \underline{0}$.

\item \textbf{Vettori linearmente indipendenti}. \linebreak 
	  Un insieme di vettori $v_1, ..., v_n$ si dice linearmente indipendente se esistono $\alpha_1, ..., 
	  \alpha_n$ tali che se $\alpha_1v_1, ..., \alpha_nv_n = \underline{0}$ allora  $\alpha_i = 0$ $
	  \forall i$.

\item \textbf{Base di uno spazio vettoriale}. \linebreak
	  Sia $V$ spazio vettoriale su $\mathbb{K}$. Allora l'insieme $B = \{v_1, ..., v_n\}$ viene definito 
	  base di $v$ se:
	
	  \begin{enumerate}
	  \item $V = Span\{v_1, ..., v_n\}$ ovvero $v_1, ..., v_n$ generano V
	  \item I vettori $v_1, ..., v_n$ sono linearmente indipendenti.
	  \end{enumerate}

\item \textbf{Teorema del completamento}. \linebreak
	  Sia $B = \{v_1, ..., v_n\}$ base di $V$ e siano $w_1, ..., w_r$ vettori linearmente indipendenti 
	  ($r\leq n$), allora esistono $n-r$ vettori di $B$ tali che insieme a $w_1, ..., w_r$ formano una 
	  base di $V$.

\item \textbf{Dimensione di uno spazio vettoriale}. \linebreak
	  Viene definita dimensione di uno spazio vettoriale finito generato $V$ il numero di vettori che 
	  compongono una qualsiasi base, e viene indicato con $dimV$.

\item \textbf{Intersezione di due spazi vettoriali}. \linebreak
	  Siano $U, W$ sottospazi vettoriali di $V$ ($U, W \leq V$) su $\mathbb{K}$ allora viene definito 
	  sottospazio intersezione il sottospazio vettoriale \hfill \break
	  $$U \cap W = \{v \in V \hspace{0.1cm}|\hspace{0.1cm} v\in U \text{\hspace{0.1cm}e\hspace{0.1cm}} v
	  \in W\}$$

\item \textbf{Somma di due spazi vettoriali}. \linebreak
	  Siano $U, W$ sottospazi vettoriali di $V$ ($U, W \leq V$) su $\mathbb{K}$ allora viene definito 
	  sottospazio somma il sottospazio vettoriale
	  $$U + W = \{u + w \hspace{0.1cm}|\hspace{0.1cm} u\in U \text{\hspace{0.1cm}e\hspace{0.1cm}} w\in W
	  \}$$
	
\item \textbf{Somma diretta di due sottospazi vettoriali}. \linebreak
	  Siano $U, W$ sottospazi vettoriali di $V$ ($U, W \leq V$) su $\mathbb{K}$ allora $V$ viene definito
	  somma diretta di $U$ e $W$ se:
	  
	  \begin{enumerate}
	  \item $V = U + W$
	  \item $U \cap W = \{0\}$
	  \end{enumerate}
	  
	  e in tal caso si indica con $V = U \oplus W$


\end{itemize}

\newpage

		     %%%%%%	Tecniche di calcolo	%%%%%
			   \section{Tecniche di calcolo}


\begin{enumerate}

\item \textbf{Verificare che un insieme è sottospazio vettoriale}. \linebreak
	  Per verificare che un insieme è sottospazio vettoriale bisogna verificare che la somma tra due 
	  vettori che appartengono all'insieme di partenza e il prodotto tra un vettore dell'insieme e uno 
	  scalare del campo $\mathbb{K}$ producano dei vettori che appartengono ancora all'insieme di 
	  partenza.


\item \textbf{Verificare che un insieme non è sottospazio vettoriale}. \linebreak
	  Per verificare che un insieme non è sottospazio vettoriale è necessario trovare due vettori che 
	  appartengono all'insieme di partenza ma la cui somma non appartiene a tale insieme, oppure un 
	  vettore che appartiene all'insieme di partenza ma che, moltiplicato per uno scalare del campo $
	  \mathbb{K}$ non appartiene più a tale insieme. Questi vengono anche chiamati controesempi.
	

\item \textbf{Verificare se una serie di vettori sono linearmente (in)dipendenti}. \linebreak
	  Per verificare se una serie di vettori dati sono linearmente dipendenti o meno, è necessario andare
	  a moltiplicarli per dei coefficienti $\alpha, \beta, ..., \omega$, quindi sommarli tra loro e porre
	  la somma uguale al vettore nullo $\underline{0}$. Risolvendo il sistema di equazioni, si
	  determinerà dunque se la soluzione dovesse essere $\alpha = \beta = ... = \omega = 0$ (in tal caso
	  i vettori saranno linearmente indipendenti) oppure no. \linebreak
	  \break
	  
	  \textbf{Esempio} \linebreak
	  Verificare che i seguenti vettori siano linearmente indipendenti: \linebreak
	  $v_1 = \begin{bmatrix}1\\5\\7\end{bmatrix}$ \hspace{1cm} $v_2 = \begin{bmatrix}1\\3\\4\end{bmatrix}
	  $ \hfill \break
	  \linebreak
	  \linebreak
	  \textit{Soluzione} \linebreak
	  Prendiamo i nostri vettori, e scriviamoli come combinazione lineare moltiplicandoli per due
	  coefficienti $\alpha$ e $\beta$, ponendo il tutto uguale al vettore nullo $\underline{0}$.
	  \linebreak \break
	  $\alpha \begin{bmatrix}1\\5\\7\end{bmatrix} + \beta \begin{bmatrix}1\\3\\4\end{bmatrix} =
	  \begin{bmatrix} 0\\0\\0\end{bmatrix}$ \hfill \break \break
	  Ora svolgiamo le opportune moltiplicazioni, e scriviamo il sistema finale: \linebreak \break
	  $\begin{cases}\alpha\cdot 1+\beta\cdot 1 = 0\\ \alpha\cdot 5 + \beta\cdot 3 = 0 \\ \alpha\cdot 7 +
	  \beta\cdot 4 =0 \end{cases} \Rightarrow \hspace{0.5cm}\begin{cases}\alpha+\beta = 0\\ 5\alpha + 
	  3\beta = 0 \\ 7\alpha + 4\beta =0 \end{cases} \Rightarrow \hspace{0.5cm} \begin{cases}\alpha=-\beta
	  \\ -5\beta + 3\beta = 0 \\ -7\beta + 4\beta =0 \end{cases} \Rightarrow \hspace{0.5cm} \begin{cases}
	  \alpha =0 \\ \beta=0 \end{cases}$ \hfill \break \break
	  Come si è visto, dunque, i due vettori sono linearmente indipendenti.
	
\end{enumerate}



%/////////////////////////////////////////////////////////////////////////////////////////////////////////



\newpage

%%%%%%%%%%%% SECONDO CAPITOLO - MATRICI %%%%%%%%%%%%
				\chapter{Matrici}

			%%%%%%	Teoria di base	%%%%%
			   \section{Teoria di base}

\begin{itemize}

\item \textbf{Matrice trasposta}. \linebreak
	  Sia $A\in M_{n,n}(\mathbb{K)}$ una matrice, viene definita matrice trasposta di $A$ la matrice 	
	  $$A^{T}=(\overline{a}_{i,j}) \text{\hspace{0.1cm} dove \hspace{0.1cm}}\overline{a}_{i,j} = a_{j,i} 
	  \hspace{0.1cm} \forall \hspace{0.1cm}i, j$$ \hfill 
	  \break ovvero la matrice che ha come righe le colonne di $A$.
	
\item \textbf{Proprietà delle matrici trasposte} \linebreak
	  Siano $A,B\in M_{n,n}(\mathbb{K)}$ e $\lambda\in \mathbb{K}$. Allora:
	  
	  \begin{enumerate}
	  \item $(\lambda A)^T=\lambda(A^T)$
	  \item $(A+B)^T=A^T+B^T$
	  \item $(A^T)^T=A$
	  \item $(A\cdot B)^T = B^T\cdot A^T$
	  \end{enumerate}
	
\item \textbf{Matrici simmetriche/antisimmetriche}. \linebreak
	  Sia $A\in M_{n,n}(\mathbb{K)}$ allora $A$ viene definita matrice simmetrica se:
	  $$A^T=A$$
	  mentre viene definita antisimmetrica se:
	  $$A^T=-A$$
	
\item \textbf{Proprietà del prodotto tra matrici} \linebreak
	  Siano $A,B, C\in M_{n,n}(\mathbb{K)}$ e $\lambda\in \mathbb{K}$. Allora:
	  
	  \begin{enumerate}
	  \item $(AB)C = (AB)C$ e $\lambda(AB) = (\lambda A)B$
	  \item $AI_n  = I_nA = A$ con $I_n$ = matrice identità di dimensione n
	  \item $A(B+C) = AB + AC$
	  \item $A\underline{0}=\underline{0}A=\underline{0}$
	  \end{enumerate}
	
\item \textbf{Matrice inversa}. \linebreak
	  Sia $A \in M_{n,n}(\mathbb{K)}$, essa è invertibile se esiste $B\in M_{n,n}(\mathbb{K)}$ tale che:
	  $$AB=BA=I_n$$
	  e, in tal caso, $B$ viene detta inversa di $A$ e si indica con $B=A^{-1}$
	
\item \textbf{Matrici simili}. \linebreak
	  Due matrici $A,B \in M_{n,n}(\mathbb{K)}$ vengono dette simili se esiste $C\in M_{n,n}(\mathbb{K)}$
	  tale che: $$A=C^{-1}BC$$
	
\item \textbf{Il determinante}. \linebreak
	  Sia $A \in M_{n,n}(\mathbb{K)}$, essa è invertibile se $detA \neq 0$.
	
\item \textbf{Matrice $A_{i, k}$}.
	  Sia $A \in M_{n,n}(\mathbb{K)}$, la matrice $A_{i, j}$ è la matrice ottenuta da $A$ cancellando la 
	  riga i-esima e la colonna j-esima. Inoltre $A_{i, j} \in M_{n-1,n-1}(\mathbb{K)}$.
	
\item \textbf{Cofattore}.
	  Viene definito cofattore lo scalare dato dalla formula: $$c_{i, j} = (-1)^{i+j}det(A_{i, i})$$
	


\end{itemize}

\newpage

		     %%%%%%	Tecniche di calcolo	%%%%%
			  \section{Tecniche di calcolo}
			   

\begin{enumerate}

\item \textbf{Calcolare il prodotto tra due matrici}. \linebreak
	  Siano $A,B \in M_{n,n}(\mathbb{K)}$ matrici: \hfill \break
	  $\begin{bmatrix}a_{1,1} & a_{1, 2} \\a_{2,1} & a_{2, 2} \\\end{bmatrix}$ \hspace{2cm}
	  $\begin{bmatrix}b_{1,1} & b_{1, 2} \\b_{2,1} & b_{2, 2} \\\end{bmatrix}$ \hfill \break
	 allora la matrice moltiplicazione sarà la matrice:
	 $$\begin{bmatrix}a_{1,1}\cdot b_{1, 1} + a_{1, 2}\cdot b_{2, 1} & a_{1, 1}\cdot b_{1, 2} + a_{1, 2}
	 \cdot b_{2, 2}\\a_{2_1}\cdot b_{1, 1} + a_{2, 2}\cdot b_{2, 1} & a_{2, 1}\cdot b_{1, 2}+a_{2, 2}
	 \cdot b_{2, 2} \\\end{bmatrix}$$
	
\item \textbf{Calcolare la matrice inversa di una matrice $A$}. \linebreak
	  Sia data $A \in M_{n,n}(\mathbb{K)}$, per calcolare la matrice inversa $A^{-1}$ è necessario scrivere
	  la matrice $[A|I_n]$ e dunque, tramite operazioni elementari sulle righe, ridurre la matrice $[A|I_n]
	  $ a matrice $[I_n|B]$, dove $B$ sarà uguale ad $A^{-1}$. \hfill \break
	  \linebreak
	  \textbf{Esempio} \linebreak
	  Sia $A=\begin{bmatrix}1 & 2 \\ 3& 7   \end{bmatrix}$. Calcolare $A^{-1}$. \hfill \break
	
	  \textit{Soluzione} \linebreak
	  Scriviamo $[A|I_n]$ e ricodunciamola ad $[I_n|A]$.
	  $$[A|I_n] = \begin{bmatrix}1 & 2 & | & 1 & 0 \\ 3& 7 & | & 0 & 1   \end{bmatrix} \leadsto 
	  \begin{bmatrix}1 & 0 & | & 7 & -2 \\ 0& 1 & | & -3 & 1   \end{bmatrix}$$
	  A questo punto la matrice $B=\begin{bmatrix}7 & -2 \\ -3 & 1   \end{bmatrix}$ sarà equivalente ad $A^
	  {-1}$ e lo si verifica facendo il prodotto $A^{-1}\cdot A$ che dovrà essere uguale ad $I_n$.
	  $$A^{-1}A = \begin{bmatrix}7 & -2 \\-3 & 1   \end{bmatrix} \cdot \begin{bmatrix}1 & 2 \\ 3& 7   
	  \end{bmatrix} = \begin{bmatrix}1 & 0 \\ 0 & 1   \end{bmatrix}$$

\item \textbf{Calcolo del determinante di una matrice}.
	  Sia $A \in M_{n,n}(\mathbb{K)}$, per calcolare il determinante di $A$ è necessario effettuare lo
	  sviluppo di Laplace rispetto una qualsiasi riga o colonna di $A$, ovvero: 
	  $$detA = a_{i, 1}c_{i, 1} + ... + a_{i, n}c_{i, n}$$
	  oppure
	  $$detA = a_{1, j}c_{1, j} + ... + a_{n, j}c{n, j}$$
	  
\end{enumerate}

%/////////////////////////////////////////////////////////////////////////////////////////////////////////

\newpage

%%%%%%%%%%%% TERZO CAPITOLO - APPLICAZIONI LINEARI %%%%%%%%%%%%
				\chapter{Applicazioni lineari}

			%%%%%%	Teoria di base	%%%%%
			   \section{Teoria di base}
			
\begin{itemize}

\item \textbf{Teorema di struttura delle soluzioni di un sistema omogeneo}. \linebreak
	  Sia $X_0\in\mathbb{K}$ una soluzione particolare del sistema lineare $Ax=b$ di ordine w, con
	  $A\in M_{n, n}\in\mathbb{K}$.
	  Allora ogni altra soluzione del sistema si scrive come $v=x_0+w$ dove $w$ è una soluzione di $Ax
	  = \underline{0}$; ovvero se $W=\{w\in\mathbb{K}^n$ | $Aw=\underline{0}\}$ le soluzioni di $Ax=b$
	  saranno: \linebreak
	  $$x_0+W:=\{x_0+w\hspace{0.1cm} |\hspace{0.1cm} w\in W\}$$
	
\item \textbf{Mappa}. \linebreak
	  Sia $A\in M_{n,n}(\mathbb{K)}$, viene definita mappa l'applicazione lineare: \hfill \break
	  $$L_A : \underset{\underline{x}}{\mathbb{K}^n} \underset{\longmapsto}{\rightarrow}
	  \underset{Ax}{\mathbb{K}^n}$$
	
\item \textbf{Applicazione lineare}. \linebreak
	  Siano $V, W$ spazi vettoriali. Viene definita applicazione (o funzione) lineare tra $V$ e $W$
	  la mappa $L:V\rightarrow W$ su $\mathbb{K}$ tale che:
	  
	  \begin{enumerate}
	  \item $L(v_i+w_i) = L(v_i) + L(w_i)$\hspace{0.5cm} $\forall v_i, w_i \in V$
	  \item $L(\lambda v) = \lambda L(v)$\hspace{0.5cm} $\forall \lambda \in \mathbb{K}$ e $v\in V$
	  \end{enumerate}
	
\item \textbf{Nucleo di una applicazione lineare}. \linebreak
	  Sia $T:V\rightarrow W$ un'applicazione lineare. Viene definito nucleo (o kernel) di $T$ l'insieme
	  dei vettori di uno spazio vettoriale su $\mathbb{K}$ tale che $T(v)=0$. Ovvero
	  $$Ker T=\{v\in V\hspace{0.1cm} |\hspace{0.1cm} T(v)=0\}$$
	
\item \textbf{Immagine di un'applicazione lineare}. \linebreak
	  Sia $T:V\rightarrow W$ un'applicazione lineare. Viene definita immagine di $T$ l'insieme
	  delle applicazioni
	  $$Im T=\{T(v)\hspace{0.1cm} |\hspace{0.1cm} v\in V\}$$ \hfill \break
	  \linebreak
	  Inoltre se $B=\{v_1, ..., v_n\}$ è base di V, allora $ImT =Span(T(v_1), ..., T(v_n))$
	
\item \textbf{Rango di un'applicazione lineare}. \linebreak
	  Viene definito rango di un'applicazione lineare $T:V\rightarrow W$ la dimensione dell'immagine
	  di T. \hfill
	  $$rgT=dim(ImT)$$
	
\item \textbf{Teorema della dimensione}. \linebreak
	  Sia $T:V\rightarrow W$ un'applicazione lineare. Allora: $$dimV=dim(KerT) + rgT$$
	
\item \textbf{Corollario del Teorema della dimensione}. \linebreak
	  Sia $T:V\rightarrow W$ un'applicazione lineare. Allora:
	  
	  \begin{enumerate}
	  \item $T$ è iniettiva $\Leftrightarrow rgT = dimV$
	  \item $T$ è suriettiva $\Leftrightarrow rgT = dimW$
	  \item Se $dimV=dimW$ ovvero $V0=W$ allora $T$ è iniettiva $\Leftrightarrow$ è suriettiva
	  \end{enumerate}
	
\item \textbf{Teorema di Rouché-Capelli}. \linebreak
	  Sia $Ax=b$ un sistema lineare di $m$ equazioni in $n$ incognite. Sia $A\in M_{n, n}(\mathbb{K})$ e
	  sia $[A|b]$ la matrice completa. Allora:
	  
	  \begin{enumerate}
	  \item Il sistema ammette soluzione $\Leftrightarrow rgA=rg[A|b]$
	  \item La soluzione, se esiste, è unica $\Leftrightarrow rgA = n$
	  \end{enumerate}
	
\item \textbf{Composizione di due applicazioni lineari}. \linebreak
	  Siano $T:U\rightarrow V$ e $T:V\rightarrow W$ applicazioni lineari di spazi vettoriali su $\mathbb{K}
	  $.
	  \hfill \break Viene definita composizione di S e T l'applicazione lineare:
	  $$T\circ S : \underset{\underline{u}}{U} \underset{\longmapsto}{\rightarrow}
	  \underset{T(S(u))}{W}$$
	
\item \textbf{Inversa}. \linebreak
	  Sia $T:V\rightarrow W$ applicazione lineare invertibile tale che esiste $S:W\rightarrow V$ lineare 
	  tale che: $$Id_V = S\circ T = T\circ S = Id_W$$ allora $S$ viene detta inversa di $T$ e si indica con 
	  $S= T^{-1}$

\end{itemize}


\newpage

		     %%%%%%	Tecniche di calcolo	%%%%%
			  \section{Tecniche di calcolo}
			  
\begin{enumerate}
\item \textbf{Verificare che un'applicazione è lineare}. \linebreak
	  Per verificare che un un'applicazione è lineare bisogna verificare che l'applicazione alla somma di 
	  due vettori coincide con la somma delle applicazioni dei singoli vettori. Inoltre bisogna verificare 
	  che l'applicazione di un vettore moltiplicato per uno scalare del campo $\mathbb{K}$ coincida con la 
	  moltiplicazione di tale scalare per l'applicazione dello stesso vettore. \linebreak
	  Ovvero sia $T:V\rightarrow W$ un'applicazione lineare e sia $\lambda \in \mathbb{K}$ uno scalare, 
	  allora:
	  
	  \begin{enumerate}
	  \item $T(v_1+v_2) = T(v_1) + T(v_2)$
	  \item $T(\lambda v) = \lambda T(v)$
	  \end{enumerate}


\item \textbf{Calcolare la composizione di due applicazioni lineari}. \linebreak
	  Per calcolare la composta di due applicazioni lineari $T$ ed $S$ è necessario scrivere le matrici
	  associate a $T$ ed $S$ e moltiplicarle tra loro.
	
\end{enumerate}


%/////////////////////////////////////////////////////////////////////////////////////////////////////////


\newpage
%%%%%%%%%%%% QUARTO CAPITOLO - SISTEMI LINEARI A SCALA %%%%%%%%%%%%
				 \chapter{Sistemi lineari a scala}

				  %%%%%%	Teoria di base	%%%%%
			   		  \section{Teoria di base}
			   
\begin{itemize}

\item \textbf{Matrice a scala}. \linebreak
	  Viene definita matrice a scala $m\times n$ a coefficienti in $\mathbb{K}$ la matrice:
	  $$\begin{bmatrix}
	  0 & 0 & p_1 & * & * & * & * \\
	  . & . & 0 & 0 & 0 & p_2 & * \\
	  . & . & . & . & . & 0 & 0 \\
  	  . & . & . & . & . & . & . & \hspace{1cm} p_n \\
	  . & . & . & . & . & . & . & \hspace{1cm} 0 & . & . & . & 0 \\
	  . & . & . & . & . & . & . & \hspace{1cm} . & . & . & . & . \\
	  . & . & . & . & . & . & . & \hspace{1cm} . & . & . & . & . \\
	  . & . & . & . & . & . & . & \hspace{1cm} . & . & . & . & . \\
	  0 & 0 & 0 & 0 & 0 & 0 & 0 & \hspace{1cm} 0 & . & . & . & 0 \\
	  \end{bmatrix}$$
	  Inoltre i termini nei posti $p_1, ..., p_n$ vengono detti $pivot$, e sono elementi diversi da $0$ con 
	  $0$ sotto e a sinistra di essi.
	
\item \textbf{Sistema lineare a scala}. \linebreak
	  Viene definito sistema lineare a scala un sistema la cui matrice (incompleta) dei coefficienti è a 
	  scala.
	
\item \textbf{Proprietà delle matrici a scala}. \linebreak
	  Sia $S\in M_{m, n}(\mathbb{K})$ una matrice a scala di pivot $p_1, ..., p_r$ tutti diversi da $0$ 
	  situati nelle colonne $j_1, ..., j_r$. Allora:
	  
	  \begin{enumerate}
	  \item $ImS = Span\{e_1, ..., e_r\}$
	  \item $rgS = r$
	  \item Una base di $ImS$ è $\left\{S^{j_1}, ..., S^{j_r}\right\}$, ovvero le colonne della matrice
	  	    che contengono i pivot.
	  \end{enumerate}

\end{itemize}
	
	
\newpage
		     %%%%%%	Tecniche di calcolo	%%%%%
			  \section{Tecniche di calcolo}
			  
\begin{enumerate}

\item \textbf{Ridurre una matrice a scala}. \linebreak
	  Per ridurre una matrice a scalare è necessario effettuare una serie di operazioni elementari sulle 
	  righe della matrice, al fine di eliminare gli elementi sottostanti e a sinistra dei pivot portandoli 
	  a 0. Le operazioni consentite sulle righe sono le seguenti:
	  
	  \begin{enumerate}
	  \item Scambio di righe: $R_i \leftrightarrow R_j$
	  \item Moltiplicazione di una riga per uno scalare diverso da 0: $R_i \rightarrow \lambda R_i$
	  \item Somma di una riga ad un'altra $R_i \rightarrow R_i+R_j$
	  \item Somma di una riga moltiplicata per uno scalare (anche =0) ad un'altra $R_i \rightarrow \lambda 
	  		R_j + R_i$
	  \end{enumerate}
	
\item \textbf{Risolvere un sistema lineare}. \linebreak
	  Per risolvere un sistema lineare nella forma $Ax=b$ è necessario:
	  \begin{enumerate}
	  \item Ridurre a scala la matrice associata a tale sistema $[A|b]$.
	  \item Una volta ridotto a scala il sistema avrà soluzione $\Leftrightarrow rg[S|c] = rgS$. \hfill
	        \break Inoltre se $rgS=rg[S|c]=n$ allora la soluzione sarà unica, altrimenti se $rgS=rg[S|c]<n$
	        esisteranno infinite soluzioni. In altre parole:
	        
	        \begin{itemize}
	  		\item Se $rgS \neq rg[S|c] \Rightarrow \nexists $ alcuna soluzione
	  		\item Se $rgS=rg[S|c]=n \Rightarrow dim(KerS) = n-rgS = n-n = 0 \Rightarrow \exists $! 
	  			  soluzione	
	  		\item Se $rgS=rg[S|c]<n \Rightarrow dim(KerS) = n-rgS > 0 \Rightarrow \exists$ $\infty$ 
	  		      soluzioni
			\end{itemize}
			
	  \end{enumerate}
	
\item \textbf{Trovare rango e base dell'immagine di uno $Span$} \linebreak
	  Per trovare il rango e la base dell'immagine di uno $Span{(v_1, ..., v_n)}$ è necessario:
	  
	  \begin{enumerate}
	  \item Scrivere la matrice associata $Ax=b$ e ridurla a scala trovando la matrice $Sx=c$.
	  \item Identificare le colonne di $A$ corrispondenti alle colonne di $S$ nelle quali si trovano i 
	    	pivot.
	  \item Base di $ImA = (A^i, ..., A^j)$ dove $i...j$ sono gli indici delle colonne di $S$ nelle quali 
	    	sono presenti i pivot.
	  \end{enumerate}
	
\item \textbf{Base e dimensione del nucleo di una applicazione}. \linebreak
	  Per trovare la base e la dimensione del nucleo di una applicazione è necessario:
	  
	  \begin{enumerate}
	  \item Ridurre a scala la matrice $A$ e contare il numero di pivot. A questo punto $rgA =$ numero 
	  	  	pivot.
	  \item Per trovare la dimenione del $KerA$ è necessario fare $dim(KerA)=n-rgA=n-rgA$.
	  \item Per trovare la base del $KerA$ è necessario risolvere il sistema omogeneo associato $Ax=
	   		\underline{0}$ e, se vi è soluzione, la base di $KerA =(v_1, ..., v_r)$ dove $KerA=\{t_1v_1+...
	  		+t_rv_r \hspace{0.1cm} | \hspace{0.1cm} t_1, ..., t_r \in \mathbb{K}\}$
	  \end{enumerate}

\item \textbf{Trovare la base e dimensione di uno Spazio Vettoriale Generato ($Span$)}. \linebreak
	  Per trovare la base di uno $Span$ di una  serie di vettori $v_1, ..., v_n$ è necessario porli come 
	  colonne di una matrice $A$ e dunque applicare il procedimento per trovare rango e base dell'immagine
	  di uno $Span$ visto nel punto 3.
	
\item \textbf{Completare un insieme $\{v_1, ..., v_r\}$ di vettori linearmente indipendenti in $\mathbb{K}
			  ^n$}. \linebreak
	  Siano $v_1, ..., v_r \in \mathbb{K}^n$ linearmente indipendenti. Allora esistono vettori di $e_1, 
	  ..., e_n$ (base canonica) tali che $\mathbb{K} = Span(v_1, ..., v_r, e_1, ..., e_{n-r})$.\hfill 
	  \break Posso dunque estrarre una base con il metodo del punto 5.

\end{enumerate}


%/////////////////////////////////////////////////////////////////////////////////////////////////////////


\newpage
%%%%%%%%%%%% QUINTO CAPITOLO - MATRICI DIAGONALIZZABILI %%%%%%%%%%%%
				\chapter{Matrici diagonalizzabili}


				  %%%%%%	Teoria di base	%%%%%
			   		  \section{Teoria di base}

\begin{itemize}

\item \textbf{Autovettori e autovalori}. \linebreak
	  Sia $T:V\rightarrow W$ un'applicazione lineare di $V$ spazio vettoriale su $\mathbb{K}$, e sia 
	  $v_0\in V$. L'elemento $v_0$ viene definito autovettore per $T$ relativo all'autovalore $\lambda \in 
	  \mathbb{K} $ se:
	  
	  \begin{enumerate}
	  \item $v_0 \neq \underline{0}$
	  \item $T(v_0) = \lambda v_0$
	  \end{enumerate}
	
\item \textbf{Autospazio}. \linebreak
	  Sia $T:V\rightarrow W$ un'applicazione lineare e sia $\lambda$ un autovalore per $T$, allora viene 
	  definito autospazio relativo a $\lambda$ il seguente insieme:
	  $$V_{\lambda}=\{v\in V \hspace{0.1cm}|\hspace{0.1cm} T(v) = \lambda V\}$$
	
\item \textbf{Matrice diagonalizzabile}. \linebreak
	  Sia $A\in M_{n,n}(\mathbb{K)}$, essa viene detta diagonalizzabile se e solo se è simile alla matrice
	  diagonale, ovvero se esiste $P\in M_{n,n}(\mathbb{K)}$ tale che:
	  $$P^{-1}AP = D$$
	
\item \textbf{Polinomio caratteristico}. \linebreak
	  Sia $T:V\rightarrow V$ una trasformazione lineare, $B=\{v_1, ..., v_n\} base di V$ e $A\in M_{n,n}
	  (\mathbb{K)}$ la matrice associata a $T$ rispetto la base $B$, allora:
	  
	  \begin{enumerate}
	  \item La funzione $p_T:\mathbb{K} \rightarrow \mathbb{K}$ viene detta polinomio caratteristico ed
	    	equivale a $p_T(x) = det(A-xI_n)$
	  \item Il polinomio caratteristico $p_T(x)$ è un polinomio di grado $n=dimV$ in $x$
	  \item Dato $\lambda\in\mathbb{K}$, questo è autovalore per $T \Leftrightarrow p_T(\lambda)=0$
	  \end{enumerate}
	
\item \textbf{Molteplicità algebrica e geometrica}. \linebreak
	  Sia $T:V\rightarrow V$ una trasformazione lineare e $\lambda\in\mathbb{K}$ un autovalore per $T$.
	  Viene definita molteplicità algebri di $\lambda$, se esiste, $n$ tale che $(x-\lambda)^n$ divide 
	  $p_T(x)$, ma $(x-\lambda)^{n+1}$ non divide $p_T(x)$, e si indica con $$m_a(\lambda) = n$$
	  Inoltre viene definita molteplicità geometrica di $\lambda$ come $$m_g(\lambda) = dimV_{\lambda}$$
	
\item \textbf{Teorema}. \linebreak
	  Sia $T:V\rightarrow V$ una trasformazione lineare di $V$ spazio vettoriale su $\mathbb{K}$. Allora 
	  sono equivalenti le seguenti affermazioni:
	  
	  \begin{enumerate}
	  \item $T$ è diagonalizzabile su $\mathbb{K}$
	  \item $p_T(x)$ ha $n = dimV$ radici contate con molteplicità in $\mathbb{K}$ e, per ogni radice $
	  		\lambda$ $m_a(\lambda) = m_g(\lambda)$
	  \item La somma delle molteplicità geometriche è $n = dimV$
	  \end{enumerate}
	
\item \textbf{Teorema}. \linebreak
	  Sia $T:V\rightarrow V$ una trasformazione lineare, questa è diagonalizzabile se e solo se esistono $n
	  $ autovettori $v_1, ..., v_n$ tali che $P[v_1 ... v_n]$ è invertibile e, in tal caso $$P^{-1}AP =
	  \begin{bmatrix}
	  \lambda_1 \\
	    & \lambda_2\\
	    &   &  .\\
	    &   &   &   .\\
	    &   &   &   &   .\\
	    &   &   &   &   & \lambda_n \\
	  \end{bmatrix}$$
	  dove $\lambda_i$ è autovalore associato a $v_i$.

\end{itemize}


\newpage
		     %%%%%%	Tecniche di calcolo	%%%%%
			  \section{Tecniche di calcolo}
			  
\begin{enumerate}
\item \textbf{Calcolo degli autovalori}. \linebreak
	  Sia $A\in M_{n, n}(\mathbb{K})$, per determinare gli autovalori è necessario trovare il polinomio
	  caratteristico tramite la formula $$p_T(x) = det(A-xI_n)$$ e dunque trovare i valori che annullano 
	  tale polinomio. Essi saranno gli autovalori della matrice $A$.
	
\item \textbf{Calcolo degli autovettori associati ad un autovalore}. \linebreak
	  Sia $A\in M_{n, n}(\mathbb{K})$ e sia $\lambda\in\mathbb{K}$ un autovalore di $A$, per trovare gli
	  autovettori associati a $\lambda$ è necessario calcolare la base di $V_{\lambda}$ dove
	  $$V_{\lambda} = Ker(A-\lambda I_n) = \{\text{soluzioni del sistema } A-\lambda I_n = 0\}$$
	
\item \textbf{Calcolo della matrice diagonalizzante $P$ e diagonale $D$}. \linebreak
	  Sia $A\in M_{n, n}(\mathbb{K})$ e siano $v_1, ..., v_n$ gli autovettori associati agli autovalori
	  $\lambda_1, ..., \lambda_n$.\hfill \break
	  La matrice diagonalizzante $P$ sarà uguale a: $$P = [v_1\text{ }...\text{ }v_n]$$
	  mentre la matrice diagonale $D= P^{-1}AP$ sarà uguale a
	  $$D=\begin{bmatrix}
	  \lambda_1 \\
	    & \lambda_2\\
	    &   &  .\\
	    &   &   &   .\\
	    &   &   &   &   .\\
	    &   &   &   &   & \lambda_n \\
	  \end{bmatrix}$$
	
\end{enumerate}





%/////////////////////////////////////////////////////////////////////////////////////////////////////////


\newpage
%%%%%%%%%%%% QUINTO CAPITOLO - PRODOTTI SCALARI %%%%%%%%%%%%
		\chapter{Prodotti scalari su $\mathbb{R}^n$}


			  %%%%%%	Teoria di base	%%%%%
			   	 \section{Teoria di base}
			   	 
\begin{itemize}

\item \textbf{Prodotto scalare (canonico).} \linebreak
	  Viene definito prodotto scalare (canonico) di due vettori $v$ e $w \in\mathbb{R}^n$ il prodotto del
	  primo vettore trasposto per il secondo:
	  $$<v, w> = v^Tw$$
	  Ovvero $$<\cdot, \cdot>:\underset{(v, w)}{\mathbb{R}^n\times \mathbb{R}^n} \underset{\mapsto}
	  {\rightarrow} \underset{v^Tw}{\mathbb{R}}$$
	
\item \textbf{Norma di un vettore}. \linebreak
	  Viene definita norma (lunghezza) di un vettore $v\in\mathbb{R}^n$ come:
	  $$||v|| = \sqrt{v^Tv}=\sqrt{<v, v>}$$
	
\item \textbf{Distanza tra due vettori}. \linebreak
	  Viene definita distanza tra due vettori $v, w\in\mathbb{R}^n$ come: $$d(v, w) = ||v-w||$$
	
\item \textbf{Proprietà del prodotto scalare}. \linebreak
	  \begin{enumerate}
	  \item Linearità rispetto le variabili. \linebreak
		    Siano $v, v_1, v_2, w, w_1, w_2\in\mathbb{R}^n$ e $\lambda\in \mathbb{R}$. Allora: \hfill 
		    \break
		    $<v_1+v_2, w> = <v_1, w> + <v_2, w>$ \hfill \break
		    $<v, w_1+w_2> = <v, w_1> + <v, w_2>$ \hfill \break
		    $<\lambda v, w> = \lambda <v, w>$ \hfill \break
		    $<v, \lambda w> = \lambda <v, w>$
	  \item Simmetria. \linebreak
		    Siano $v, w\in\mathbb{R}^n$, allora: \hfill \break
		    $<v, w> = <w, v>$
	  \end{enumerate}
	
\item \textbf{Angolo tra due vettori}. \linebreak
	  Viene definito angolo tra due vettori $v, w\in\mathbb{R}^n$ non nulli come l'angolo $\theta\in[0, 
	  \pi]$ tale che: $$cos\theta = \frac{<v, w>}{||v||\cdot||w||}$$
	
\item \textbf{Ortogonalità di due vettori}. \linebreak
	  Due vettori vengono detti ortogonali (perpendicolari) tra loro se $$<v, w> = 0$$ e, in tal caso, si 
	  indica con $v\perp w$
	
\item \textbf{Base ortogonale}. \linebreak
	  Viene definita base ortogonale di $\mathbb{R}^n$ una base composta da vettori a due a due ortogonali.
	
\item \textbf{Versore}.
	  Viene definito versore un vettore la cui norma (lunghezza) è pari a $1$.
	
\item \textbf{Base ortonormale}. \linebreak
	  Viene definita base ortonormale di $\mathbb{R}^n$ una base ortogonale i cui elementi hanno tutti 
	  norma (lunghezza) pari a $1$ (sono tutti versori).
	
\item \textbf{Osservazione}. \linebreak
	  Siano $v_1, ..., v_n$ vettori di una base ortonormale. Allora $$<v_i, v_j> = \delta_{ij}$$ dove
	  $$\delta_{ij}=\begin{cases}
	  1 \hspace{0.2cm} i=j \\
	  0 \hspace{0.2cm} i\neq j\\
	  \end{cases}$$
	
\item \textbf{Osservazione}. \linebreak
	  Sia $v\in\mathbb{R}^n$ un vettore, $v\neq\underline{0}$. Allora $\frac{v}{||v||}$ ha norma $1$.
	
\item \textbf{Teorema}. \linebreak
	  Sia $V\leq \mathbb{R}^n$ e $v_1, ..., v_n$ generatori di $V$ con $v_i\neq\underline{0}\forall i$.
	  Allora esistono $w_1, ..., w_r$ generatori di $V$ tali che:
	  
	  \begin{enumerate}
	  \item $Span(w_1, ..., w_j) = Span(v_1, ..., v_j) \hspace{0.2cm} \forall j\leq r$
	  \item $w_j\perp Span(w_1, ..., w_{j-1}) \hspace{0.2cm} \forall j=2, ..., r $
	  \item $<w_j, v_j> = 0 \hspace{0.2cm} \forall j$
	  \end{enumerate}	
	
\item \textbf{Proposizione}. \linebreak
	  Sia $V$ spazio vettoriale $\leq\mathbb{R}^n$ e sia $U\leq V$. Allora $\forall v\in V$ $\exists !$
	  $u_0\in U$ tale che: $$v-u_0 \perp u$$
	
\item \textbf{Proiezione ortogonale}. \linebreak
	  Sia $U\leq V\leq \mathbb{R}^n$. Viene definita proiezione ortogonale di $v$ su $u$ la mappa lineare
	  $$p_u = proj_u: \underset{v}{V} \underset{\mapsto}{\rightarrow} \underset{u_0 = p_u(v)}{U} $$
	  tale che:
	  
	  \begin{enumerate}
	  \item $proj_u(v) \in U$
	  \item $v-proj_u(v) \perp U$
	  \end{enumerate}
	  
	  Inoltre $proj_u(v) = \sum\limits_{i=1}^r \frac{<v, u_i>}{<u_i, u_i>}\cdot u_i$
	
\item \textbf{Complemento ortogonale}. \linebreak
	  Sia $S\subseteq \mathbb{R}^n$. Viene definito $$S^{\perp} = \{v\in\mathbb{R}^n \hspace{0.1cm} |
	  \hspace{0.1cm} v\perp S\}$$
	  Se $S=V\leq\mathbb{R}^n$ allora $V^{\perp}$ viene definito complemento ortogonale di $V$ in $
	  \mathbb{R} ^n$
	
\item \textbf{Proposizione}. \linebreak
	  Sia $U\leq \mathbb{R}^n$, allora:
	  
	  \begin{enumerate}
	  \item $\mathbb{R}^n = U\oplus U^{\perp}$
	  \item $dimU^{\perp} = n-dimU$
	  \item $\forall v\in\mathbb{R}^n$ $\exists !$ $v_{\parallel}\in U$ e $v_{\perp}\in U_{\perp}$ tale che
		    $v = v_{\parallel} + v_{\perp}$
	  \item $||v|| = ||v_{\parallel}|| + ||v_{\perp}||$
	  \end{enumerate}
	

\end{itemize}


\newpage
		     %%%%%%	Tecniche di calcolo	%%%%%
			  \section{Tecniche di calcolo}
			  
\begin{enumerate}

\item \textbf{Calcolare la base ortogonale di uno $Span$}. \linebreak
	  Dato $V = Span(v_1, ..., v_n)$ per calcolare una sua base ortogonale è necessario applicare il 
	  procedimento di ortogonalizzazione di Gram-Schmidt che consiste in due passaggi:
	 
	  \begin{enumerate}
	  \item Porre il primo vettore $v_1 = w_1$
	  \item Calcolare il vettore j-esimo tramite la formula
		    $$w_j = v_j - \sum_{k=1}^{j-1}\left(\frac{<v_j, w_k>}{<w_k, w_k>}\right) $$
	  \end{enumerate}

\item \textbf{Calcolare la base ortonormale di uno $Span$}. \linebreak
	  Dato $V = Span(v_1, ..., v_n)$ per calcolare una sua base ortonormale è necessario prima calcolare la
	  base ortogonale tramite il procedimento del punto 1 e, successivamente, normalizzare tutti i vettori
	  dividendoli per la propria norma.
	
\item \textbf{Calcolare dimensione e base del complemento ortogonale di uno $Span$}. \linebreak
	  Dato $U = Span(v_1, ..., v_n)$ per calcolare la dimensione di $U^{\perp}$ bisogna applicare la 
	  formula: $$dimU^{\perp} = dimV - rgA$$
	  dove $$dimV = n \text{ e } A=\begin{bmatrix}
	  v_1 & v_2 & ... & v_n\\
	  \end{bmatrix}$$
	  Per calcolare la base di $U^{\perp}$ bisogna invece risolvere il sistema omogeneo $Cx=0$ dove
	  $$C = \begin{bmatrix}
	  v_1^T \\ \\ v_2^T \\ . \\ . \\ . \\  \\ v_n^T
	  \end{bmatrix}$$
	 
\item \textbf{Calcola la proiezione di un vettore su uno $Span$}. \linebreak
	  Sia $w \in\mathbb{R}^n$ un vettore, e sia $V = Span\{v_1, ..., v_n\}$. Per calcolare la proiezione di
	  $w$ su $V$ è necessario applicare la seguente formula:
	  $$proj_V(w) = \sum_{k=1}^{n}\left(\frac{<v_k, w>}{<v_k, v_k>}\right)$$
	
\item \textbf{Calcolare le coordinate di un vettore rispetto la base di uno $Span$}. \linebreak
	  Sia $w \in\mathbb{R}^n$ un vettore, e sia $V = Span\{v_1, ..., v_n\}$. Per calcolare le coordinate di
	  $w$ rispetto la base $v_1, ..., v_n$ di $V$ è necessario:
	  
	  \begin{enumerate}
	  \item Applicare la formula per il calcolo di $proj_V(w)$.
	  \item Gli scalari $\frac{<v_k, w>}{<v_k, v_k>}$ con $1\leq k\leq n$ saranno le coordinate rispetto a
	  $v_1, ..., v_n$, ovvero rispetto la base di $V$, di $w$.
	  \end{enumerate}

\end{enumerate}




%/////////////////////////////////////////////////////////////////////////////////////////////////////////


\newpage
%%%%%%%%%%%% SESTO CAPITOLO - MATRICI ORTOGONALI %%%%%%%%%%%%
				\chapter{Matrici ortogonali}
				

			  %%%%%%	Teoria di base	%%%%%
			   	 \section{Teoria di base}
			   	 
\begin{itemize}

\item \textbf{Matrice ortogonale}. \linebreak
	  Sia $P\in M_{n, n}(\mathbb{R})$ una matrice, essa viene detta ortogonale se:
	  
	  \begin{enumerate}
	  \item è invertibile
	  \item $P^{-1} = P^T$ ovvero $PP^T = In = P^TP$
	  \end{enumerate}
	
\item \textbf{Teorema}. \linebreak
	  Sia $P\in M_{n, n}(\mathbb{R})$, essa è ortogonale se e solo se le colonne di $P$ sono base 
	  ortonormale di $\mathbb{R}^n$, ovvero se le righe di $P$ sono base ortonormale di $\mathbb{R}^n$.
	
\item \textbf{Isometria}. \linebreak
	  Sia $T: \mathbb{R}^n \rightarrow \mathbb{R}^n$ una trasformazione lineare, essa viene definita 
	  isometria se: $$<T(v_1, v_2)> = <v_1, v_2>$$
	
\item \textbf{Matrice ortogonalmente diagonalizzabile}. \linebreak
	  Sia $A\in M_{n, n}(\mathbb{R})$, essa viene definita ortogonalmente diagonalizzabile se esiste $P$
	  tale che:
	  
	  \begin{enumerate}
	  \item $P$ è ortogonale
	  \item $P^{-1}AP (= P^TAP)$ è diagonale
	  \end{enumerate}
	
\item \textbf{Corollario del Teorema degli Assi Principali}. \linebreak
	  Sia $A\in M_{n, n}(\mathbb{R})$, allora essa è ortogonalmente diagonalizzabile se e solo se è 
	  simmetrica.
	
\item \textbf{Teorema Spetrale}. \linebreak
	  Sia $T: \mathbb{R}^n \rightarrow \mathbb{R}^n$ un'applicazione lineare. Allora $T$ è simmetrica se e
	  solo se è ortogonalmente diagonalizzabile.
	
\item \textbf{Lemma Teorema Spetrale}. \linebreak
	  Sia $A\in M_{n, n}(\mathbb{R})$ e siano $v_1, ..., v_r$ autovettori associati agli autovalori 
	  distinti $\lambda_1, ..., \lambda_r$. Allora $v_1, ..., v_r$ sono ortogonali tra loro, ovvero $<v_i, 
	  v_j>= 0$ $\forall i\neq j$.

\end{itemize}



\newpage
		     %%%%%%	Tecniche di calcolo	%%%%%
			  \section{Tecniche di calcolo}
			  
\begin{enumerate}
\item \textbf{Trovare la matrice $P$ ortogonale e diagonalizzante una matrice $A$}. \linebreak
	  Sia $A\in M_{n, n}(\mathbb{R})$, per trovare $P$ ortogonale tale che $P^TAP=D$ è necessario:
	  \begin{enumerate}
	  \item Verificare che la matrice $A$ sia simmetrica.
	  \item Trovare gli autovalori $\lambda_1, ..., \lambda_r$ e gli autospazi associati a tali autovalori
		    $E_{\lambda_1}, ... E_{\lambda_r}$.
	  \item Per ogni autospazio $E_{\lambda_i}$ trovare una base ortogonale.
	  \item Per ogni autospazio $E_{\lambda_i}$ trovare una base ortonormale (normalizzare i vettori del 
		    punto (c) dividendo ogni vettore $w_i$ per la sua norma $<w_i, w_i> = ||w_i||$).
	  \item Unire le basi ortonormali degli autospazi, ovvero i vettori $\frac{w_i}{||w_i||}$. La loro
		    unione sarà base ortonormale di $\mathbb{R}^n$ e quindi i vettori $\frac{w_i}{||w_i||}$ saranno
		    le colonne di $P$, ovvero $$ P=\begin{bmatrix}
		    \mathlarger{\frac{w_1}{||w_1||}} & . & . & . & \mathlarger{\frac{w_r}{||w_r||}} \end{bmatrix}$$
	  \end{enumerate}

\end{enumerate}



\end{document}
